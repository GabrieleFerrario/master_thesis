\chapter{Sviluppi Futuri}
In questo capitolo viene fornita una lista dei lavori futuri che vengono considerati utili per migliorare questo progetto:
\begin{enumerate}
    \item Implementazione di un modello di Image Segmentation costruito appositamente per task che prevedono coppie immagine-testo. Questo modello può essere progettato seguendo tre strategie differenti:
    \begin{enumerate}
        \item Instance Segmentation: il modello viene costruito sfruttando i dataset di Instance Segmentation più appropriati per questi task, attualmente i dataset disponibili più interessanti risultato: \acrshort{coco} \cite{lin2014microsoft}, \acrshort{lvis} \cite{gupta2019lvis}, ADE20K \cite{zhou2017scene} (prestando attenzione agli oggetti di tipo stuff) e OpenImages \cite{OpenImages}. Tramite questi dataset non vengono catturati gli oggetti di tipo stuff poichè risultano più difficili da identificare per un modello di Instance Segmentation. Pertanto, si dovrebbe utilizzare un modello che potrebbe essere in grado di localizzare questi oggetti, il quale considera anche il dataset \acrshort{coco} Stuff. Un modello molto interessante è \acrshort{mask_rcnn}, poichè è una variante di \acrshort{faster_rcnn} che si è comportata molto bene con questi oggetti nel task di Object Detection.
        Infine, il modello può essere reso robusto sfruttando tecniche di data augmentation molto interessanti come Copy-Paste\footnote{\url{https://github.com/tensorflow/tpu/tree/master/models/official/detection/projects/copy_paste}} \cite{ghiasi2020simple};
        \item Panoptic Segmentation: viene definito un nuovo dataset costruito dall'unione tra \acrshort{coco} Panoptic e \acrshort{lvis}, ridefinendo le maschere di segmentazione affinché non ci siano oggetti sovrapposti. Il dataset ottenuto viene utilizzato per allenare un modello di Panoptic Segmentation in grado di generare delle segmentazioni della scena ancora più ricche di oggetti rilevati. Infine, può essere utilizzato anche il dataset ADE20k che prevede sia oggetti di tipo stuff che thing;
        \item Combinazione tra Object Detection e Image Segmentation: il modello viene costruito tramite un paradigma di addestramento parzialmente supervisionato \cite{hu2018learning}. Questi paradigmi permettono di addestrare modelli di segmentazione su un grande insieme di classi che hanno tutte annotazioni di bounding box, ma solo una piccola frazione di esse ha le annotazioni delle maschere di segmentazione. Questa strategia risulta molto interessante perché i dataset di segmentazione consentono di riconoscere meno classi rispetto a quelle che si potrebbero rilevare tramite i dataset di Object Detection. Il modello utilizzato dovrebbe  decomporre il problema della segmentazione nei sottoprocessi di rilevamento degli oggetti tramite bounding box e previsione della maschera;
    \end{enumerate}
    \item Retrain di \acrshort{oscar}$_+$ su triple che considerano anche gli oggetti segmentati;
    \item Sviluppare modelli linguistici leggeri pre-addestrati su molti dataset che prevedono coppie immagine-testo, poiché quando si hanno poche risorse computazionali come in questa tesi sono richiesti molti giorni per il fine-tuning. Infatti, i fine-tuning eseguiti in questa tesi hanno richiesto più di sei giorni ed è stata usata la versione costruita su \acrshort{bert} base. Questi modelli più leggeri possono essere ottenuti tramite Knowledge Distillation \cite{fang2021compressing} e recentemente è stato sviluppato il modello MiniVLM \cite{wang2020minivlm}, del quale non è stato ancora reso disponibile il modello pre-addestrato;
    \item Migliorare la rappresentazione degli oggetti considerando le relazioni e le interazioni;
    \item Testare tecniche di data augmentation sui dataset di Image Captioning, sia sulle immagini che sulle didascalie;
\end{enumerate}